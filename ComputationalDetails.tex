\section {Computational Details}

Classical MD simulations of succinic acid solutions were performed in neat-water aqueous slabs. A water cube of side-length 30 \angs was packed with 900 \wat, and 4 succinic acid molecules to reach a nominal concentration of 0.25 M succinic acid. Initial placement of the molecules was done by randomly packing them into the cubic dimensions with a minimum inter-molecular packing distance of 2.1 \angs.\cite{Martinez2009, Martinez2003} The final cell dimensions of 30x30x100 \angs$^3$ were produced by a subsequent elongation of the cubic unit cell along the z-axis. Periodic boundary and constant-volume conditions resulted in a unit cell in an ``infinite slab'' configuration, with two air/water interfaces, both perpendicular to the z-axis.\cite{Wilson1991}

The particle mesh ewald (PME) summation procedure was used for calculation of long range electrostatic interactions, and non-bonded interactions were cut off at 10 \angs.\cite{Essmann1995} Atomic overlaps and energetically unfavorable contacts were minimized using conjugate gradient and steppest descent minimization procedures. Subsequent equilibration was done by evolving the system for 2 ns. A production run was performed for 10 ns of simulation using 1 fs timesteps. Atomic coordinate snapshots were recorded every 100 fs for a total of 100,000 datapoints. Simulations were carried out at 300 K in the NVT ensemble. The SHAKE algorithm was used to constrain all water OH bonds.\cite{Ryckaert1977}

The Amber 11 molecular dynamics suite was used for performing the MD.\cite{Case2010} Water molecules wre modeled using the POL3 polarizable water model,\cite{Dang1992, Caldwell1995} and the succinic acid was constructed using the general amber force field (GAFF) polarizable parameter set.\cite{Wang2004} Fully polarizable models were used for \wat~and succinic acid. Polarizability has been shown to be crucial for accurately reproducing water and solute behavior near and within interfacial regions.\cite{Dang1998} Atomic partial charges were calculated for the succinic acid atoms using the RESP procedure with the R.E.D. toolset, and the Gamess quantum calculation package.\cite{Pigache2004, Dupradeau2010, Schmidt1993}
